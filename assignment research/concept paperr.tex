\documentclass[11pt]{article}

\begin{document}
\title{LIMITED INTERNET CONNECTIVITY AT COLLEGE OF COMPUTING AND INFORMATION SCIENCES (COCIS) AT MAKERERE UNIVERSITY} 
\maketitle
\section{1.	INTRODUCTION}
        Internet connectivity at Makerere University in general and COCIS in particular has greatly detoriated over the years to the extent that when you try to connect to a wireless hotspot, it displays “limited connectivity”.
The purpose of this research is to highlight some of the reasons for this unreliable network connections and the possible solutions.
                                                                                                                                                                                                                                       
\section{1.1.	BACKGROUND TO THE PROBLEM}
                  COCIS is a college that was established on 13 December 2010. It is made up of SCIT and EASLIS and is one of the largest computing and information computing training, information science, research and consultancy colleges in Africa. It has a state of art infrastructure including theater lectures, giant computer laboratories, specialized computer laboratories and a college library.
The iLab Internet was started at MIT laboratories in 1998 by professor Jesus Del Alamo and in 2005 it was introduced in Makerere through MIT that sought to establish links for utilization of iLabs in Africa with support from The Carnegie Corporation of New York in order to enrich science and engineering education and was headed by Mr. Albert Lumu, a developer at the university then.
Previously, internet within the university especially at the COCIS computer laboratories has been fast due to the low student population and few student devices connected to it but of recent years with the student population peaking close to 40000 at the university and about 2000 at the college, there has been decreasing low bandwidth due to increase in internet enabled devices owned by the students which has greatly affected the quality of output at the college in as far as access to online resources is concerned.
\section{1.2.	PROBLEM STATEMENT}
       Every student who joins the university is mandated to pay Technology development fee of USH 50000 that is meant to cater for technology related issues such as paying the internet service providers and servicing of computers and yet the students do not really get value for their money as stated because if there is no internet then you cannot use the computers.
      As the topic states, limited internet connectivity at COCIS is indeed a huge challenge in as far as the success of students through online research is concerned and there is need to dig deeper into this problem and analyze it in order to clearly demulsify the problem and come up with everlasting solutions for it.

\section{1.3.	AIMS AND OBJECTIVES}
      The aims are both general and specific.

\section{1.3.1.	GENERAL OBJECTIVES}
1.	To identify the reasons for the limited internet connectivity at the university.
2.	To come up with possible solutions to the above problem.

\section{1.3.2.	SPECIFIC OBJECTIVES}
1.	To identify the impact of limited internet to the students and lecturers at COCIS
2.	To find out what positive changes are most likely to be noticed among the students if the problem is resolved.
         
\section{1.4.	RESEARCH SCOPE}
       The research is being conducted within Makerere University in general but with closer attention to COCIS which is most likely to limit the authenticity of the research since there is scanty information about the state of internet connectivity in other colleges and areas within the university.
     The research is fully centered on the technology (internet) sector and therefore leaving out other sectors such as the health sector, welfare sector, sports sector, to mention but a few.

\section{1.5.	RESEARCH SIGNIFICANCE}
     The students, lecturers, public, and the university in general are the shareholders of this research and all are to benefit accordingly because the students will not only be able to excel in the academics through prompt and efficient research but also the lecturers and other stake holders are set to eat big from the ‘’CAKE’’.

     
\section{2.	METHODOLOGY}
       The type of research that will be used in this study is descriptive and qualitative research. 
Qualitative research investigates ‘why ‘and ‘how’ of decision making. A variety of tools are to be used to gather information such as the use of questionnaires, photographs to capture images and videos, oral interviews, and most importantly the use of ODK collect and through sampling method. Correspondents are to be selected randomly from the college and within university and asked questions such as their satisfaction in as far as the university internet connectivity is concerned and the changes they would propose to make it a better system.


\section{REFERENCES}
1.	www.linkedin.com
2.	New Vision newspaper of 26th July 2007
3.	www.merriam-webster.com
4.	www.internetsociety.com



\end{document}